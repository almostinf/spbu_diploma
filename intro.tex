\specialsection{Введение}

В настоящее время системы видеоконференций являются неотъемлемой частью повседневной коммуникации. Развитие технологии связи привели к появлению различных платформ и технологии, обеспечивающих возможность удаленного общения. Среди них немаловажное место занимает фреймворк WebRTC (Web Real-Time Communication), который позволяет осуществлять передачу аудио и видеоданных в реальном времени через веб-браузер без необходимости установки дополнительного программного обеспечения.

В рамках данной научной исследовательской работы мы рассмотрим историю развития систем видеоконференций, обоснуем выбор WebRTC в контексте современных требований и ожиданий пользователей. Кроме того, мы рассмотрим основные проблемы, с которыми сталкиваются разработчики систем видеоконференций и варианты их решения.

\pagebreak