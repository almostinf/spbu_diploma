\specialsection{ЗАКЛЮЧЕНИЕ}

В рамках этой научно-исследовательской работы мы рассмотрели историю развития видеоконференций, начиная с их зарождения в 1920-ых до современных технологий. Подробно рассмотрели протокол WebRTC и его возможности для обеспечения качественной коммуникации между участниками видеоконференций. Определили ожидания конечных пользователей и проблемы, влияющие на качество видеоконференций, а также рассмотрели, как WebRTC справляется с задачей передачи видео.

Одной из ключевых тем исследования было масштабирование участников видеоконференций. В частности, мы рассмотрели проблемы, связанные с ограничениями P2P архитектуры WebRTС и альтернативы, позволяющие решить проблему с масштабированием.

В будущих исследованиях мы подробно рассмотрим MCU (Multiple Control Unit) архитектуру для WebRTC, а также сосредоточимся на разработке и оптимизации приложения, устойчивого к условиям постоянного увеличения числа участников в видеоконференции.

\pagebreak