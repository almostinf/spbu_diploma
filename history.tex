\specialsection{ВВЕДЕНИЕ}

В настоящее время системы видеоконференций являются неотъемлемой частью повседневной коммуникации. Развитие технологии связи привели к появлению различных платформ и технологий, обеспечивающих возможность удаленного общения. Среди них немаловажное место занимает фреймворк WebRTC (Web Real-Time Communication), который позволяет осуществлять передачу аудио и видеоданных в реальном времени через веб-браузер без необходимости установки дополнительного программного обеспечения.

В рамках данной научной исследовательской работы мы рассмотрим историю развития систем видеоконференций, обоснуем выбор WebRTC в контексте современных требований и ожиданий пользователей. Кроме того, мы рассмотрим основные проблемы, с которыми сталкиваются разработчики систем видеоконференций и варианты их решения.

Необходимость проведение научно-исследовательской работы обусловлена ростом популярности систем видеоконференций и технологическими вызовами, которые они ставят перед их разработчикам. Исследование и усовершенствование таких систем позволит удовлетворить потребности конечных потребителей в качественной и надежной видеосвязи.

\pagebreak