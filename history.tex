\section{История}

История систем видеоконференций началась в 1920-ых годах, когда появились первые стабильные и работающие телекамеры, тогда же компания AT&T Bell Telephone Laboratories создала работающий комплекс телефонной связи, которые транслировал изображение на расстояние 200 миль. В 1930-ых та же компания AT&T продемонстрировала сеанс двусторонней видеосвязи между офисами AT&T на Манхэттене, однако затянувшиеся последствия Великой депрессии затормозили развитие видеосвязи.

В 1980-ых компания Compression Labs составила конкуренцию AT&T и выпутила CLI T1 в качестве первой коммерческой системы групповой видеоконфенцсвязи. Ее первоначальная стоимость составляла 250000 долларов, а каждый звонок стоил 1000 долларов в час.

В 1990-ых произошел Бум Интернета и развития цифровой телефонии. В 1991 году студенты факультета компьютерных наук Кембриджского университета изобрели первую веб-камеру.

В начале 2000-ых появились смартфоны, оснащенные камерами на задней и передней панелях для съемки фотографий. Три эстонских инженера-программиста представили Skype в августе 2003 года. Два бывших сотрудника Yahoo основали WhatsApp в 2009 году как приложение для мгновенного обмена сообщениями. В 2020-ых блокировка COVID-19 заставило многих людей по всему миру работать из дома, что привело к буму на все видеоустройства и большой популярности систем видеоконференций \cite{v1}.

\pagebreak